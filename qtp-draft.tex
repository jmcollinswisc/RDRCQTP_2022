
\documentclass[12pt]{article}
\usepackage[margin=1in]{geometry}
\usepackage[english]{babel}
\usepackage[utf8x]{inputenc}
\usepackage{amsmath}
\usepackage{amsfonts}
\usepackage{amssymb}
\usepackage{parskip}
\usepackage{rotating}
\usepackage{natbib}
\usepackage{subfigure}
\usepackage{tikz}
\usepackage{graphicx}
\usepackage{watermark}
\thiswatermark{\centering \put(-70,-749){\includegraphics[scale=1]{cover.png}} }
\usepackage{titlesec}
\usepackage{xcolor}
\titleformat*{\section}{\filcenter\Large\bfseries\sffamily\color{red}}
\titleformat*{\subsection}{\large\bfseries\sffamily\color{red}}
\titleformat*{\subsubsection}{\sffamily\color{red}}
\usepackage{fancyhdr}
\pagestyle{fancy}
\usepackage{pdfpages}
\usepackage{graphicx} 
\usepackage{siunitx} 
 \usepackage{booktabs} 
 \usepackage{comment}
 \usepackage{longtable}
 \usepackage{hyperref}
 \usepackage{showlabels} % turn off if final version - useful to match tables to reference commands
 
%% ENTER SHORT TITLE for RUNNING HEAD
\lhead{\textsf{COVID and Disability}} % controls the left corner of the header
%%
\cfoot{} % controls the center of the footer
\rhead{\textsf{Page~\thepage}} % controls the right corner of the footer
 

\begin{document}


\thispagestyle{empty}
\cleardoublepage{}

\hspace{-2cm}\begin{minipage}[b]{0.3\linewidth}
\begin{flushleft}
 
\vspace{6cm}

%%% ENTER AUTHOR NAMES HERE   

\textsf{J.  Michael Collins} \\
\textsf{University of Wisconsin--Madison} \\
 \bigskip
\textsf{Carly Urban} \\
\textsf{University or Montana}\\
\end{flushleft}
\end{minipage}

\hspace{5cm}\begin{minipage}[t]{0.8\textwidth}
\begin{center}

%%% ENTER TITLE HERE
\Large{\textbf{\textsf{WI22-QTP: Economic Security of People with Disabiloties during the Pandemic}}}\\
\end{center}
\end{minipage}

\hspace{5cm}\begin{minipage}[t]{0.7\textwidth}
\begin{flushleft}
\vspace{6cm}

%% STANDARD DISCLAIMER
\footnotesize{\textsf{The research reported herein was performed pursuant to a grant from the U.S. Social Security Administration (SSA) funded as part of the Retirement and Disability Consortium. The opinions and conclusions expressed are solely those of the author(s) and do not represent the opinions or policy of SSA or any agency of the Federal Government. Neither the United States Government nor any agency thereof, nor any of their employees, makes any warranty, express or implied, or assumes any legal liability or responsibility for the accuracy, completeness, or usefulness of the contents of this report. Reference herein to any specific commercial product, process or service by trade name, trademark, manufacturer, or otherwise does not necessarily constitute or imply endorsement, recommendation or favoring by the United States Government or any agency thereof.}}
\end{flushleft}

\end{minipage}
 


\cleardoublepage{}


\cleardoublepage{}

% SPACING 
\linespread{1.25} 
\section*{Abstract}
This study documents the economic impact of COVID19 on households with disabilities.

\newpage 
 
\section{Introduction}

\subsection{Background}

 
 
 
There are multiple paths through which the COVID-19 pandemic may affect the economic security of working age adults with disabilities. Disruptions to the economy and risk of disease spread led some who were working prior to the pandemic to exit the labor market \citep{cheng2020back, goda2021impact moen2020disparate, quinby2021older }, thereby reducing income through wage earnings. This reduction in labor force participation has not been offset by a subsequent increase in Social Security  claiming [CITE?].   


Reduced consumption \citep{baker2020does,horvath2021covid} 

households often unable to pay their bills \citep{clark2021financial,schneider2020household}. 

  
We begin our analysis by  

We further explore mechanisms underlying changes 

 

Taken together, the results of this study provide important detail about the depth and breadth of the inequity of financial hardship experienced in the pandemic. Our findings indicate heterogeneous effects for vulnerable segments of the population.    


\section{ Prior Literature}

 

\subsection{COVID-19 and the Financial Security of Older Adults }

There is a burgeoning body of literature on the economic consequences of the COVID-19 pandemic for U.S. households. Most directly, several studies analyze changes in labor force participation and unemployment in response to the COVID-19 pandemic (e.g.  \cite{cheng2020back}), with a few studies focusing on labor trends  \citep{goda2021impact,moen2020disparate,quinby2021older}.   

Reductions in income may be offset in part by reductions in consumption  \citep{baker2020does,casado2020aggregate,chetty2020did,farrell2020consumption,horvath2021covid}. For example, \cite{farrell2020consumption} find an overall 10 percent decline in consumer spending following the onset of the pandemic. However, the effects on consumption are heterogeneous across a number of dimensions.  \cite{chetty2020did} find that much of the reduction in spending is concentrated among higher income households—households in the top income quartile spent 13 percent less as of mid-July 2020 relative to January 2020, whereas households in the bottom income quartile reduced consumption by only 4 percent during the same period. For those experiencing a COVID related loss of income, \cite{farrell2020consumption} find that receipt of pandemic-related unemployment benefits is associated with a 10 percent increase in consumer spending relative to the prior year. 


 

 
 
\section{Data and Methods}


\subsection{Summary Statistics}


\subsection{OLS Specifications}
We provide descriptive regression analysis, following the form:

$\delta outcome_{i} = \alpha_0 + \beta_1*\texttt{Post}_i + \beta_2*\texttt{STUFF}_i + \epsilon_i $

 

\section{Findings}

summary tables

regressions

Coef plots
 

 
  

\section{Conclusion}
 
Implications for Social Security Programs
 
Increased eligibility for SSI and Medicaid, as well as SNAP and LIHEAP
 

\section{Figures}
 
 

\begin{figure}[!ht]\label{stuff}
\caption{stuff}
\centering
\includegraphics[scale=0.65]{stuff.png}
\medskip 
\begin{minipage}{0.65\textwidth} 
{\footnotesize Here are some notes that go with the graph.  \par}
\end{minipage}
\end{figure}

\newpage
 
\section{Tables}
 

 

\begin{table}[!h]
\caption{Stuff}
\label{tab:newtable}
 \scalebox{0.55}{   
\begin{tabular}{ll}
\hline
All                     & 50.2 \\ \hline
A   household & 48.1 \\
B   household   & 50.2 \\
Primary   Person White  & 42.9 \\
Primary   Person Black  & 61.3 \\
Primary   Person Latina & 66.8 \\ \hline
\end{tabular}  }
\end{table} 
 
  



 
\clearpage


\section*{References}
\renewcommand*{\refname}{\vspace*{-12mm}} 

%%  bib file with citations saved as "refs.bib"
\bibliography{refs}
\bibliographystyle{chicago}

\section*{Appendix}
\clearpage
 
\includepdf[scale=1]{back.pdf}
 

\end{document}


